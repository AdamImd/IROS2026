% \documentclass[]{ieeeconf} 
\documentclass[letterpaper, 10 pt, conference]{ieeeconf} 
\IEEEoverridecommandlockouts
\overrideIEEEmargins
%\pdfobjcompresslevel=0
%\pdfminorversion=4


% \PassOptionsToPackage{draft}{graphicx} 
\usepackage{graphicx}
\usepackage{caption}
\usepackage{hyperref}
\usepackage{amssymb}
\usepackage{amsmath}
\usepackage{amsfonts}
% \usepackage[shortlabels]{enumitem} 

\usepackage{algorithm}
\usepackage{algpseudocode}
\usepackage{xcolor}

\usepackage{cite}
% Using BibTeX with IEEEtran style

\newcommand{\acro}{{CLOAK}}
\newcommand{\todo}[1]{{\color{red} TODO:#1}}
% \newcommand{\ktxt}[1]{{\color{blue} #1}}
\newcommand{\ktxt}[1]{}

\title{\LARGE \bf \acro: Contact Sensing with Low-Cost Force-Sensing Skin and Contact-Aware Inverse Kinematics
% (LoFiSkin)
}
% \author{Anonymous authors}
\author{Adam Imdieke, Heidi Woelfle, Minghao Zou, Brad Holschuh and Karthik Desingh}


\let\oldtwocolumn\twocolumn
\renewcommand\twocolumn[1][]{%
    \oldtwocolumn[{#1}{
        \centering
        % \includegraphics[width=0.340\textwidth]{images/Teaser.png}
        % \includegraphics[width=0.60\textwidth]{images/Spot.png}
        \includegraphics[width=0.95\textwidth]{images/Teaser_both.pdf}
        \captionof{figure}{\acro{} enables whole-arm tactile sensing by utilizing 3D printed TPU shells with embedded force-sensitive resistor arrays. Contact-aware inverse kinematics allows the robot to both embrace and avoid contact during manipulation.}\label{fig:intro}}]
}



\begin{document}
\maketitle
\thispagestyle{empty}
\pagestyle{empty}


\begin{abstract}
Robotic manipulation in human-centered and cluttered environments requires not only dexterous control, but also the ability to sense and regulate contact forces across the entire arm. Fingertip and hand sensing alone are insufficient; whole-arm tactile sensing is needed to avoid or  embrace or avoid contact during manipulation. However, building effective tactile skins remains challenging due to multiple design trade-offs, including fabrication and assembly cost, surface coverage on existing manipulators in the market, taxel density, sensing capabilities beyond binary contact detection, and long-term durability and repeatability. 
In this work, we present \acro{}, a low-cost, 3D-printed whole-arm skin with fabric surface layers and a high-density force-sensitive resistor array. We describe key design choices that balance cost, coverage, and sensitivity. Complementing the hardware, we introduce Contact-IK, a custom inverse kinematics algorithm that integrates tactile feedback to enable manipulation strategies where contact can be both embraced and avoided. We demonstrate \acro{} and Contact-IK on the Boston Dynamics Spot, highlighting the advantages of whole-arm tactile sensing in manipulation tasks. For more details and results, refer to the project website: \url{https://rpm-lab-umn.github.io/CLOAK_Webpage/}
\end{abstract} 


\section{Introduction}
Humans naturally exploit whole-arm contact during manipulation, using surfaces to brace, stabilize, or guide movements in cluttered environments. Recent advances in robotic tactile sensing have demonstrated the value of extending sensing beyond grippers to enable safer physical human-robot interaction~\cite{10610014}, whole-body contact awareness~\cite{8812712,kohlbrenner2025gentact}, and contact-rich manipulation strategies~\cite{8593555,10.7551/mitpress/9481.003.0014}. Despite these advances, widespread adoption remains limited: most robotic systems still lack tactile coverage beyond the hands, limiting their ability to safely and effectively exploit environmental contacts as humans do~\cite{6502183,Cyskin}.

While progress has been made, several challenges continue to hinder broader adoption of whole-arm tactile sensing in robotics. Existing research prototypes that extend tactile sensing to the forearm or whole arm~\cite{6502183,Cyskin,kohlbrenner2024machine} often rely on hardware that is costly, fragile, or difficult to replicate, creating a gap between demonstration systems and practical deployment.
% Moreover, even when tactile data is available, integrating it effectively into control algorithms to enhance safety and performance remains an active challenge~\cite{8276298,9616379,ashkanazy2023collision,10375143}. 
Bridging these gaps—developing accessible, robust tactile hardware and control methods that leverage whole-arm contact—is essential for enabling robots to exploit whole-body contact as humans do.

To bridge these gaps, we introduce \acro{}, a low-cost and robust tactile skin that can be readily integrated into diverse robot embodiments. \acro{} builds on \cite{FSR_WTL} and uses 3D-printed inner and outer shells that form a high-coverage array of force-sensitive resistors (FSRs), generated directly from the arm's 3D mesh to conform to complex geometries. To leverage this sensing capability, we develop Contact-IK, a contact-aware inverse kinematics algorithm that incorporates tactile feedback into null-space motion. This enables robots to regulate contact forces during manipulation—avoiding excessive collisions while proactively embracing beneficial contacts. For mobile manipulators with redundant degrees of freedom, Contact-IK exploits the additional null space to simultaneously achieve end-effector goals while managing whole-body contact.

The contributions of this paper are as follows:
\begin{itemize}
    \item A low-cost, whole-arm tactile skin (\acro{}) consisting of modules with high-coverage FSR arrays built with fabric or TPU surface layers, designed directly from robot mesh models to conform to complex arm geometries.
    \item Fabrication guidelines and sensor characterization demonstrating the practical feasibility and performance of the proposed tactile skin design.
    % \item Contact-IK, a contact-aware inverse kinematics algorithm that incorporates tactile feedback into null-space optimization, enabling both contact avoidance and contact embrace strategies.
    \item Experimental validation on a mobile manipulator platform demonstrating \ktxt{TODO}.
\end{itemize}

\section{Related Work}
\label{sec:related_works}


\begin{figure*}[t]
\centering
\includegraphics[width=0.95\textwidth]{images/CLOAK_Overall.pdf}
\caption{Illustration of the tactile sensor design. 1) showing the sensor on the Spot Robot arm, 2) showing the inner and outer layers of the sensor along with their ridges, 3) showing the 3D structure of the taxel array, 4) showing the 2D cross-section before and during contact, 5) showing the layer structure of each taxel.} 
\label{fig:rigid_tactile_sensor}
\end{figure*}


%Tactile sensing has long been recognized as a critical capability for dexterous manipulation and safe physical interaction. Most tactile systems primarily focus on sensing at the end effector, limiting the robot's ability to exploit contact across its entire body \ref{TODO}. 
% Here we discuss recent advancements in whole-arm/body tactile sensing mechanisms and contact-aware motion planning and control along with differentiable techniques in inverse kinematics with contact-awareness. 

Recent advances in whole-arm tactile sensing allow robots to perceive and interpret contact across larger surface areas, significantly improving manipulation capabilities and interaction safety. In this section, we review related work in three areas central to our contributions: a) whole-arm tactile sensing, b) contact-aware motion planning, and c) differentiable inverse kinematics (IK).

\subsection{Whole-Arm Tactile Sensing}
% Kohlbrenner et al.~\cite{kohlbrenner2025gentact} proposed a method for creating large-area tactile skins using 3D printing and conductive materials. Their design emphasizes ease of fabrication and integration, making it accessible for various robotic platforms. The Gentact skin provides high coverage and adjustable density, but low sensitivity due to its reliance on capacitive sensing. Because the system uses RC delay sensing, the total number of sensing nodes in skin is limited, which reduces the spatial resolution. In their implementation, each skin unit contains 6 nodules at most, resulting in a total of 36 sensing nodes. The capacitive sensing approach also limits the ability to detect fine-grained contact information, such as force magnitude and direction, and requires careful calibration to avoid interference from environmental factors. Capacitive sensing also limits the contact objects to be conductive, which is not always the case in many real-world scenarios. We build upon their easily fabricated snap on design, but we use a resistive sensing approach to achieve higher spatial resolution and sensitivity.

% Kohlbrenner et al.~\cite{kohlbrenner2024machine} previously proposed a machine learning approach to localize contact in variable density three-dimensional tactile artificial skin. Their method uses a neural network to predict contact locations based on a mutual capacitance sensor and wires in silicone rubber on a 3D printed base. While their approach demonstrates the potential for low cost tactile sensing, it does not provide a complete solution for whole-arm tactile sensing, especially in terms of robustness to environmental factors and difficult geometry. Our work extends this by providing a more robust resistive sensing approach that can be easily integrated onto existing robotic platforms, without requiring extensive calibration or machine learning models.
Kohlbrenner et al.~\cite{kohlbrenner2025gentact} introduced GenTact, a large-area tactile skin fabricated with 3D printing and conductive materials. The design emphasizes ease of fabrication and integration, offering high coverage and adjustable density across robotic platforms. However, its reliance on capacitive RC-delay sensing limits sensitivity and spatial resolution—each unit contains only six nodules (36 total)—and restricts detection of fine-grained contact such as force magnitude and direction. It is also susceptible to environmental interference and requires conductive objects for contact. Building on their snap-on design, we instead use a resistive sensing approach to achieve higher spatial resolution, greater sensitivity, and broader applicability. 

In earlier work, Kohlbrenner et al.~\cite{kohlbrenner2024machine} proposed a machine learning method to localize contact in variable-density 3D tactile skins using mutual capacitance sensors in silicone rubber on a 3D-printed base. While this demonstrated the potential of low-cost tactile sensing, it lacked robustness to environmental factors and complex geometries. Our approach extends this direction with a resistive sensing strategy that integrates easily onto existing robotic platforms without requiring extensive calibration or learning models.

Compton et al.~\cite{FSR_WTL} demonstrated a textile FSR array using Eeonyx piezoresistive fabric between knit electrodes, achieving a low-cost, flexible tactile sensor. While effective for on-body sensing, their design is less suited for robotic arms due to challenges in conforming to complex geometries and ensuring durability under repeated manipulation tasks. Our approach builds upon that by using flexible 3D-printed shells with FSR arrays adhered, providing a robust solution tailored for robotic applications.


% https://www.tum.de/en/news-and-events/all-news/press-releases/details/35732-1
Cheng et al.~\cite{8812712} developed an artificial whole-body skin that combines multimodal sensing with high coverage, but its fabrication is costly and complex. In contrast, our approach emphasizes ease of integration—requiring only 3D models of robot links—while providing high spatial resolution and sensitivity at lower cost. Similarly, CySkin~\cite{Cyskin} offers whole-arm tactile sensing with capacitive sensors, delivering high coverage, sensitivity, and density. Though effective, it is available only as a proprietary package, making it expensive and less customizable. Our approach offers a more affordable and customizable solution that can be tailored to specific robotic platforms and applications.

Other notable efforts in large-scale tactile sensing with flexible taxel arrays include Maiolino et al.~\cite{6502183} and CushSense by Xu et al.~\cite{10610014}. While these systems demonstrate promising coverage and scalable fabrication, deploying tactile skins on real manipulation platforms typically also requires low cost, straightforward mechanical/electrical integration, and reliable performance when wrapped over complex link geometries and used repeatedly. \acro{} is designed to meet these practical constraints through a low-cost resistive taxel array and a fabrication workflow that can be adapted to existing robot link models.


\subsection{Contact-Aware Motion Planning and Control}

In motion planning and control, contact has traditionally been treated as a constraint to avoid, with planners generating trajectories that steer clear of obstacles and potential contact points~\cite{Giftthaler17112017}. More recent research has shown the benefits of embracing contact for improved stability and manipulation. In legged locomotion, planners that model contact dynamics demonstrate greater robustness to disturbances and terrain variations~\cite{8276298}. In manipulation, exploiting contact through pushing, pivoting, and environmental bracing enables execution of more complex tasks~\cite{10.7551/mitpress/9481.003.0014, 8593555}.

These approaches typically handle contact at the planning layer rather than within the kinematic optimization itself. In contrast, we integrate tactile feedback directly into the IK objective, enabling real-time contact avoidance and contact-seeking behaviors without redesigning the planner.

\subsection{Differentiable IK with Contact Awareness}
Inverse kinematics (IK) is often formulated as an optimization problem to find joint configurations that achieve a desired end-effector pose while satisfying constraints~\cite{Giftthaler17112017,zhao_article}. Such constraints enable null-space optimization for secondary objectives, including \textit{minimizing joint torques}, \textit{avoiding contact}, or \textit{maintaining stability}.

However, few IK systems incorporate real-time tactile measurements from distributed sensors, particularly for mobile manipulators with whole-arm coverage. Instead, they often rely on known obstacle locations or simplified contact models~\cite{10375143, 9616379, 9345975, ashkanazy2023collision}, which limits adaptability in dynamic, unstructured, or partially observable environments.

Our approach formulates ContactIK as minimizing a differentiable objective that includes pose error, joint-limit barriers, and tactile contact terms. We use JAX automatic differentiation to compute gradients of this objective and apply quasi-Newton updates, enabling the IK solver to reach an $SE(3)$ end-effector target while adapting online to contact sensed by the tactile skin.







\section{Sensor Design}



We propose a whole arm, soft force sensing matrix  optimized for 1) maximal coverage, 2) analog taxel readings and 3) low-cost and ease of fabrication. 

Each sensor matrix is composed of rows and columns of conductive material separated by a piezoresistive layer (Velostat). 
When contact is made within a taxel, the piezoresistive layer compresses, creating a change in resistance that is read by a microcontroller using a voltage divider circuit. While our sensor design is adaptable to most robot geometries, we target the Boston Dynamics Spot robot, where we design and evaluate our sensor for the arm of the robot.

We expore two variations of the sensor: 1) a 3D printed TPU shell with ????, and 2) a fabric-based sensor using conductive fabric. Both sensors are designed to be easily fabricated using commerially available materials and integrated onto existing robot platforms. The 3D printed sensor provides a more robust and durable solution, while the fabric sensor offers a more sensitive and flexible option. We describe the design and fabrication of each sensor in the following sections. An overview of the sensor designs is shown in Fig~\ref{tactile_sensors_both}.


\subsection{TPU Sensor}
To design the 3D printed inner and outer layers of the skin, we use the mesh files provided by Boston Dynamics' URDF Files, and then apply the following sequence of CAD operations on the link geometry of interest:

% \begin{figure*}
%     \centering
%     \includegraphics[width=0.95\textwidth]{images/fig_hardware.pdf}
%     \caption{TODO}
%     \label{fig:hardware}
% \end{figure*}

To design the 3D-printed skin, we start from the link-level mesh files provided by Boston Dynamics (Fig \ref{fig:tactile_sensors_both}-2 Yellow) as part of their URDF models. For each link, we apply the following sequence of CAD operations:
\begin{enumerate}
\item \textbf{Inner layer:} Generate a conformal shell with a uniform thickness of \textit{3 mm} offset outward from the robot’s base geometry  (Fig~\ref{fig:tactile_sensors_both}-2 Green).
\item \textbf{Outer layer:} Generate a second shell with a wall thickness of \textit{3 mm}, whose inner surface is offset by \textit{5 mm} from the robot’s base geometry, resulting in a gap between the two layers (Fig~\ref{fig:tactile_sensors_both}-2 Orange).
\item \textbf{Ridges:} Within the inter-shell region, add horizontal ridges on the inner shell and vertical ridges on the outer shell. The ridges are 2 mm thick, and the spacing between adjacent ridges determines the resulting taxel density (Fig~\ref{fig:tactile_sensors_both}-2 Cutout).
\item \textbf{Clearance Cuts:} Material is removed near kinematic joints to prevent interference during motion, and a cut is added to facilitate installation of the skin(Fig~\ref{fig:tactile_sensors_both}-1).
\end{enumerate}

% We then 3D print both shells with a flexible material (Thermoplastic Polyurethane - TPU) as it allows for easy snap on assembly, along with deformable contact surfaces. We use a outer thickness of \textit{3 mm}, and a ridge spacing of approximately \textit{20 mm}. 
% For the inner shell, we adhere copper tape to a Velostat sheet using it's conductive adhesive, and then add a layer of 0.15mm thick double sided adhesive tape to the copper side. The resulting Copper-Velostat sheet is then adhered to the inner shell, using a knife along the sensor ridges to remove excess Velostat, electrically isolating each row/column. The outer shell is prepared in a similar manner, where we directly adhere the copper tape to the inner surface of the shell, and then remove excess copper along the sensor ridges. Wires are connected to each row and column using copper tape, and then routed to a microcontroller. The outer shell is then snapped onto the inner shell, and secured to the robot using a layer of packing tape across the seam.

Both shells are 3D printed using 95A TPU with an outer shell thickness of \textit{3\,mm}, with ridge spacing of approximately \textit{20\,mm}.

For the inner shell, copper tape is laminated onto one side of a Velostat sheet via its conductive adhesive, followed by a double-sided adhesive layer on the copper layer. The resulting laminate stack is attached to the shell. For the outer shell, copper tape is directly applied. This structure is shown in Fig~\ref{fig:tactile_sensors_both}-5

Wires are connected to each row and column, which are then routed to a microcontroller. The assembly is completed by snapping the outer shell onto the inner shell and securing it to the robot.

If binary contact detection is sufficient for the application, the Velostat layer can be omitted, where copper tape is adhered directly to both layers. This creates a binary contact sensor, significantly reducing the cost and complexity of the sensor.


% and \textit{less than 2 hours} of human effort per link.



\subsection{Textile Sensor Array}
The textile sensor array is designed to be a more flexible and sensitive option made entirely from off-the-shelf materials. The sensor array consists of two outer fabric layers, two layers of woven conductive fabric (Silver-coated Nylon, Less EMF) and a piesoresitive layer (3M Velostat). The bottom outer layer is made from a non-slip backing cloth with high friction to prevent slipping against the robot while the top is made from a lightweight woven cotton to allow for a thinner and more sensitive surface. A woven conductive nylon fabric is cut into strips and adhered to the outer layers using a heat-activated adhesive web (Pellon 805 Wonder-Under).  Finally, the piezoresistive layer is cut into individual squares and placed at each intersection of the rows and columns, creating an array of 30 sensors. The outer layers are joined together by applying a fabric adhesive (E6000 Fabri-Fuse) in between the conductive rows and columns and joining them together. A hook and loop fastener is used to wrap the sensor array around the robot. We get similar taxel coverage to the TPU sensor, where boundaries of the taxels use the same spacing as the TPU sensor. To interface with the microcontroller, we use snap buttons to connect wires to the rows and columns of the sensor, and a bias resistor of 47$\Omega$ for the voltage divider circuit.

% \begin{enumerate}

% \end{enumerate}

% \ktxt{ADD MORE INFORMAITON HERE}

% \ktxt{VERIFY COST AND TIME}

% \ktxt{VERIFY PROCESS STEPS}


% \begin{figure}
%     \centering
%     \includegraphics[width=0.95\columnwidth]{images/Sensor_WTL.pdf}
%     \caption{1) Instron universal testing machine used for sensor characterization. 2) 7.5mm square applicator above a 0.5 mm thick outer layer taxel. 3) We characterize the sensor response of each taxel of the \acro{} sensor.}
%     \label{fig:sensor_wtl}
% \end{figure}



% \section{ContactIK: Contact-Aware Inverse Kinematics}
% Using our proposed sensor, we introduce ContactIK: a contact-aware inverse kinematics method that serves as a drop-in replacement for conventional IK routines. 


% When contact is not desired, ContactIK operates transparently and independently of the higher level end effector position controller, where given a cartesian space target, the joint space of the robot is optimized to reach the goal configuration while maintaining a secondary contact objective. 


% When contact is desired, ContactIK enables the operator to specify a desired contact location on the surface of the robot to act as a new end effector, along with a desired contact location in the environment. ContactIK then optimizes the joint space of the robot to reach the desired end-effector pose, utilizing the contact sensor to detect the magnitude of contact force, and regulate movements to prevent excessive forces. We also optimize for the original end-effector pose to be maintained as closely as possible. This allows the robot to embrace contact while still maintaining its original task objective.

% % \subsection{Gradient Descent IK}
% % Given a desired end-effector pose \( \mathbf{T} \in SE(3) \), ContactIK performs gradient descent of the current joint state \( \mathbf{J} \in \mathbb{R}^n \) and a set of contact constraints 
% % \[
% % \mathcal{C} = \{ (\mathbf{P}, \mathbf{R}, \mathbf{F}) \},
% % \]
% % where \( \mathbf{P} \in \mathbb{R}^3 \) is the contact position, \( \mathbf{R} \in SO(3) \) is the contact orientation, and \( \mathbf{F} \in \mathbb{R}^1 \) is the contact force value.

% % The gradient descent algorithm returns a joint configuration \( \mathbf{J}^* \) that minimizes positional error along with any additional contact objectives. 
% % % However, in some cases, a valid solution may not be found due to kinematic infeasibility, singularities, local minima in the gradient space. 

% % ContactIK runs at \textit{60- Hz} on the CPU for inverse kinematics only, and \textit{200 Hz} with 2 contact points. 


% \subsection{Collision Avoidance IK}
% For a robot arm with $M$ joints and joint angles $\omega_{1:M}$, we define the forward kinematics function as $FK(\omega_{1:M}): \mathbb{R}^M \rightarrow SE(3)$, that maps joint angles to an end-effector pose in $SE(3)$. 
% % The robot is underactuated if $M < 6$, and overactuated if $M > 6$. An underactuated is not guaranteed to reach every position and orientation in SE(3), while an overactuated robot has infinite solutions to reach a given position and orientation. We focus on the overactuated case, as the Boston Dynamics Spot arm has 6 revolute joints and we add two prismatic joints to control the base of the arm, leading to a total of 8 joints. Because the arm is overactuated, we can condition the inverse kinematics to optimize for other objectives, such as avoiding obstacles or embracing contact. This can be refferred to as conditioning the null space of the inverse kinematics. A null space is the set of all joint angles that do not affect the end effector position and orientation, so in our case of $M=8$ joints, the null space has a dimension of $M-6=2$.
% It can be decomposed into a series of transformations from each joint as shown in equation~\ref{eq:fk}. 
% % We define the forward kinematics function as $FK(\omega_{1:M}): \mathbb{R}^M \rightarrow SE(3)$, which can be decomposed into a series of transformations from each joint as shown in equation~\ref{eq:fk}. 
% % Our approach is agnostic to the specific forward kinematics implementation, as long as it is differentiable, allowing compatibility with most robot representations. Generally, the forward kinematics can be expressed as a series of matrix multiplications of the individual joint transformations. 
% The forward kinematics can be expressed as:
% \begin{equation}
% \label{eq:fk}
% FK(\omega_{1:M}) = \prod_{i=1}^{M} {}^{i-1}\!T_{i}(\omega_i) = T_{1}(\omega_1) T_{2}(\omega_2) \ldots T_{M}(\omega_M)
% \end{equation}

% where $T_i(\omega_i)$ is the transformation from the coordinate frame of joint $i$ given its joint angle $\omega_i$ to its next frame.

% Given the desired end-effector pose $X_d \in SE(3)$, we use a quasi-Newton method to iteratively solve for the joint angles $\omega^*_{1:M}$ that minimize the total loss function:
% \begin{equation}
% \label{eq:ik}
% \omega^*_{1:M} = \arg\min_{\omega} L_{\text{pos}} + L_{\text{rot}} + L_{\text{limits}} + L_{\text{tactile}}
% \end{equation}
% where each loss term is defined as follows:

% \subsubsection{Positional loss}
% The positional loss uses the L2 norm:
% \begin{equation}
% L_{\text{pos}} = \| p_c - p_d \|_2^2
% \end{equation}
% where $p_c$ and $p_d$ are the current and desired positions extracted from $X_c = FK(\omega)$ and $X_d$ respectively.

% \subsubsection{Rotational loss}
% The rotational loss uses the geodesic distance on $SO(3)$:
% \begin{equation}
% L_{\text{rot}} = \arccos\left(\frac{\text{trace}(R_c^T R_d) - 1}{2}\right)
% \end{equation}
% where $R_c$ and $R_d$ are the current and desired rotation matrices.

% \subsubsection{Joint Limits}
% The joint limit constraints use a logarithmic barrier function:
% \begin{equation}
% L_{\text{limits}} = \mu_{\text{barrier}} \sum_{i=1}^{M} \left(-\log(\omega_i - \omega_i^{\min}) - \log(\omega_i^{\max} - \omega_i)\right)
% \end{equation}
% where $\omega_i^{\min}$ and $\omega_i^{\max}$ are the lower and upper joint limits, and $\mu_{\text{barrier}}$ is a barrier weight parameter.

% \subsubsection{Tactile loss}
% The tactile loss $L_{\text{tactile}}$ encodes contact avoidance behavior. We store the most recently detected 3D contact point and reuse it for subsequent IK solves until a new contact is detected, at which point the stored point is overwritten. We represent tactile contact points as $\mathbf{c}_j \in \mathbb{R}^3$ for $j = 1, \ldots, N_{\text{tactile}}$ in the world frame (in our implementation, $N_{\text{tactile}}=1$). Robot surface geometry points $\mathbf{m}_i \in \mathbb{R}^3$ for $i = 1, \ldots, N_{\text{robot}}$ are the corresponding surface samples expressed in the world frame under the current joint configuration (i.e., obtained by transforming link-frame surface samples through forward kinematics). We compute:
% \begin{equation}
% L_{\text{tactile}} = \sum_{i=1}^{N_{\text{robot}}} \sum_{j=1}^{N_{\text{tactile}}} \left[\max(0, r_{\text{tact}} - \|\mathbf{m}_i - \mathbf{c}_j\|_2)\right]^2
% \end{equation}
% where $r_{\text{tact}}$ is a distance cutoff, and the distance violation is clipped to zero for separations exceeding this cutoff. This quadratic penalty encourages the robot to move geometry away from detected contact regions.
% % TODO: Optimizer, Iterations Rot distance

% % The optimization is performed continuously, updating the joint angles at each iteration. Our approach does not require an explicit Jacobian matrix or a hessian, as we use automatic differentiation to compute the gradients needed for the quasi-Newton updates. Using the automatic differentiation allows us to easily incorporate complex conditioning terms and constraints into the optimization, as long as they are differentiable.

% % \subsection{Implementation Details}
% % We implement the forward kinematics function $FK(\omega)$ using MuJoco's XML format, specifying the joint types, link lengths, and other relevant parameters. We use the \texttt{mujoco-py} library to load the model and compute the forward kinematics using PyTorch tensors to enable automatic differentiation. We define the optimization loop using PyTorch's autograd functionality to compute the gradients of the loss function with respect to the joint angles. We use the TODO optimizer from PyTorch to perform the quasi-Newton updates, as it is well-suited for problems with a moderate number of parameters and can handle non-linear objectives. Our gradient descent loop is shown in Algorithm \ref{alg:ik}.

% % \begin{algorithm}
% % \caption{Inverse Kinematics with Contact-Aware Conditioning}\label{alg:ik}
% % \begin{algorithmic}[1]
% % \Require Desired end effector pose $X_d$, initial joint angles $\omega_0$, gradient update scalar $\alpha$
% % \While {True}
% %     \State $\omega \gets \omega_0$
% %     \State $X_c \gets FK(\omega)$ \Comment{Compute current end effector pose}
% %     \State $L \gets \| X_c - X_d \|^2 + \lambda C(\omega)$ \Comment{Compute loss}
% %     \State $g \gets \nabla_\omega L$ \Comment{Compute gradient of loss}
% %     \State $\omega \gets \omega - \alpha g$ \Comment{Update joint angles}
% %     % \State \Return $\omega$ \Comment{Return optimized joint angles}
% %     % \State \textbf{wait} until next control cycle
% % \EndWhile
% % \end{algorithmic}
% % \end{algorithm}

% % The contact term $C(\omega)$ can be defined to the create desired behavior of the robot. For simple collision avoidance, upon detecting contact with the environment, we create a contact location $\mathbf{P}$ and define our loss as the proximity of the contact location on the robot to $\mathbf{P}$, pushing the robot away from the contact. 

% \subsection{Contact Embracing IK}
% When intentional contact is required, the operator specifies:
% \begin{itemize}
%     \item Robot surface pose: contact point $\mathbf{P}_r$ (on link $k$ local frame) and orientation $\mathbf{R}_r$ (relative to link $k$)
%     \item Environment contact pose: point $\mathbf{P}_e$ and orientation $\mathbf{R}_e$ (both in world frame)
% \end{itemize}

% ContactIK defines the desired contact pose using the chosen orientation and environment contact point:
% \[\mathbf{T}_{target} = \begin{bmatrix}
% \mathbf{R}_e & \mathbf{P}_e \\
% 0 & 1
% \end{bmatrix}\] 

% and a new end-effector frame is defined at the contact point on the robot:
% \[
% \mathbf{T}_{robot} =
% \underbrace{{}^{0}\!T_{1}(\omega_1)\,
% {}^{1}\!T_{2}(\omega_2)\,\cdots\,
% {}^{k-1}\!T_{k}(\omega_k)}_{\mathbf{T}_{link}}
% \cdot
% \begin{bmatrix}
% \mathbf{R}_r & \mathbf{P}_r \\
% 0 & 1
% \end{bmatrix}\]

%  The IK target moves the arm towards the target along the contact surface normal, while the original end-effector pose is also optimized to remain as close as possible to its original target. The loss function is modified as follows:
% \begin{equation}
% \label{eq:ik_contact}
% \omega^*_{1:M} = \arg\min_{\omega} L_{\text{contact}} + L_{\text{original}} + L_{\text{limits}}
% \end{equation}
% where $L_{\text{contact}}$ is the positional and rotational loss between $\mathbf{T}_{robot}$ and $\mathbf{T}_{target}$, and $L_{\text{original}}$ is the positional and rotational loss between the original end-effector pose $FK(\omega)$ and its desired target.

% Once contact is detected above a threshold force, the task is considered successful and the arm backs off slightly to maintain light contact.

\section{Results}\label{sec:results}


\begin{figure}
    \centering
    \includegraphics[width=0.95\columnwidth]{images/Samples.pdf}\\
    \includegraphics[width=0.95\columnwidth]{images/Graph_Cycle.pdf}\\
    \includegraphics[width=0.95\columnwidth]{images/Graph-05_resistance_lines.pdf}
    \caption{\textbf{Top}: Illustration of the different outer layer thicknesses tested, we test thicknesses of 0.5mm, 1mm, 2mm, 3mm, and 5mm.    \textbf{Middle}: Load and unload curves for various outer layer thicknesses. The binary contact behavior of these sensors is shown with the force at $\inf \omega$ Thinner layers have higher sensitivity, but saturate at lower forces.
    \textbf{Bottom}: Binary contact threshold forces are found at the boundary between no response ($\infty \Omega$) and measurable response. Thinner layers have lower threshold forces and higher sensitivity, while thicker layers have a larger range of detectable forces.}
    \label{fig:thickness}
\end{figure}


In this section, we characterize the behavior of \acro{}, and the performance of ContactIK in both collision avoidance and contact embracing scenarios. We use our Instron testing machine, with a reference resistor of 1000$\Omega$ in a voltage devider circuit. We test the response of the skin by applying a 7.5$\times$7.5mm square applicator at the center of each taxel, and measuring the voltage response of the taxel. We perform five load unload cycles from 0 to 30N. 

% and hybrid sensors, and 47$\Omega$ for the fully fabric sensors.
% at a rate of TODO mm/s.

\subsection{Sensor Characterization}
To optimize our design, we evaluate the response of one call of the sensor across a range of outer shell thickness. We use an Instron universal tensile tester to find the response of the sensor to a known force, where we apply forces from \textit{0-30N} to a single taxel for five load-unload cycles. We then fit a curve to the loading cycle to create a model of the sensor response.


We use a common 30$\times$30mm$\times$3mm TPU flat inner layer with 2mm ridges to characterize our outer layers. We varied the thicknesses of the outer layers across the range of $[0.5, 1, 2, 3, 5]\ $mm, and tested their response to applied forces.

% We fit a second-order polynomial to the response curve of each cell and used the fit to determine trends in sensitivity and range as a function of material thickness. For the polynomial $F(x)=ax^{2}+bx+c$, we use the coefficients to understand sensor behavior. We define low-force sensitivity $S_{\text{low}} = b$ and high-force sensitivity $S_{\text{high}} = \left. (2ax + b) \right|_{x = 15\,\text{N}}$. The ridges of our TPU also provide a binary response, as the voltage across the sensor is nearly zero until contact is made between the inner and outer layers. We define the threshold force $F_{\text{thresh}} = c$ as the force at which the sensor begins to respond.

We find that thinner TPU layers result in higher sensitivity and lower threshold forces, while thicker layers provide a larger range of detectable forces. We find that a 3mm outer shell provides a good sensitivity and range based on our application.

% \subsection{Full Matrix Characterization}
% To evaluate the performance of the full force sensing skin, we created a skin for the midsection of the fourth link of Spot's arm. We create a 6$\times$5 matrix with 35$\times$35mm taxels as shown in Fig.~\ref{fig:skin}. We apply the skin to a 3D printed version of the link.

% The 3D printed skin consists of a 3mm TPU inner layer, a 2mm ridge height, and a 3mm TPU outer layer. We collect five load-unload cycles per taxel and fit a curve to the response of each taxel. \todo{Analysis}

\subsection{Full Skin Characterization}
\begin{figure}
    \centering
    \includegraphics[width=0.95\columnwidth]{images/TPU_threshold_heatmap.pdf}
    \caption{Heatmap of the threshold forces across all taxels of the \acro{}. Many taxels have threshold forces between 2-4 N, with some outliers likely due to the complex geometry and non-uniform pressure distribution during testing.}
    \label{fig:heatmap}
\end{figure}

We then characterize the \acro{} skin applied to a 3D printed version of the fourth link of Spot's arm. The skin consists of a 6$\times$5 matrix with 20$\times$20mm taxels, with a 3mm TPU inner layer, 2mm ridge height, and 3mm TPU outer layer. We collect four load-unload cycles per taxel and fit a curve to the loading responses of each taxel. 

We find that while the threshold forces vary significantly across taxels as seen in Figure~\ref{fig:heatmap}, most likely due to the non-uniform geometries of the link. We find that the mean threshold force across all taxels is 2.78 N, with a standard deviation of 1.29 N. This indicates that the ideal conditions of the single taxel characterization do not fully translate to the more complex geometry and smaller taxel size of the full skin. However, the overall performance is sufficient for our application, and we can use the fitted curves to estimate contact forces during manipulation tasks.

% Mean: 2.78 N
% Std Dev: 1.29 N
% Min: -0.07 N
% Max: 5.03 N
% Median: 2.86 N


\subsection{Contact Avoidance}
\begin{figure}
    \centering
    \includegraphics[width=0.95\columnwidth]{images/Fig_Avoid.pdf}
    \caption{The contact avoidance objective loss term is a sum of the penetration distances between the that the avoidance points on the robot and a sphere with a radius of the contact threshold distance. This causes the robot to move away from any contacts. }
    \label{fig:avoid}
\end{figure}

Using ContactIK, we are able to optimize the joint configuration of the robot to react to collisions in real-time, avoiding obstacles while maintaining the origional end-effector objective. In figure \ref{fig:avoid}, we show an example of the robot reacting to contact with a human, while keeping the end-effector in the same pose. We find that even with a high number of surface contact points (eg. 10000) per link, ContactIK maintains a real-time performance of 30 Hz at 1000 iterations per inverse kinematics computation.



\subsection{Contact Embracing}
\begin{figure*}
    \centering
    \includegraphics[width=0.95\linewidth]{images/FIg_Embrace.pdf}
    \caption{The contact embracing task aims to balance a dustpan in the gripper, and open a door while preventing large forces. To open the door, we create a new end effector on the arm of the robot and move along its normal until a force above a threshold is achieved. We also optimize the rotation angle of the gripper to prevent it from tipping over.}
    \label{fig:embrace}
 \end{figure*}
 To evaluate the performance of ContactIK in contact embracing scenarios, we use a door that requires tactile feedback to safely open. We place the robot arm inside the doorway, and use ContactIK to gently slide the door open while keeping the gripper level with the ground. Using the tactile feedback from the \acro  sensor, we are able to detect when the door reaches its fully open position, and maintain a final contact force. This method effectively demonstrates the ability of ContactIK to utilize tactile feedback whole-body, contact-aware control for multiple objectives simultaneously.
% \section{Conclusion}

This paper introduced \acro{}, a low-cost whole-arm tactile skin, and Contact-IK, a contact-aware inverse kinematics algorithm that enables robots to both avoid and embrace contact during manipulation. By combining a practical sensing solution with differentiable optimization, we demonstrated that mobile manipulators can leverage whole-arm tactile feedback to perform contact-rich tasks in cluttered environments.

The \acro{} sensor addresses key barriers to widespread adoption of whole-arm tactile sensing. Using 3D-printed modules with FSR arrays and fabric surface layers, it provides a high coverage analog sensor with a material cost under \$10 and requiring less than two hours of assembly time per link. Our characterization showed that a 3mm TPU outer shell provides a favorable balance between sensitivity and dynamic range, with sufficient force detection threshold to suppress noise while maintaining responsiveness for manipulation tasks. The design workflow leverages robot mesh files, and enables adaptation to diverse robotic platforms without specialized fabrication equipment or extensive calibration.

Contact-IK complements this hardware by incorporating tactile feedback directly into the kinematic optimization process. Unlike traditional planners that treat contact as a binary constraint, Contact-IK operates in two modes: contact avoidance, where detected contacts generate repulsive terms in the null-space objective, and contact embrace, where the robot actively seeks and regulates contact forces while maintaining end-effector goals. Our experiments demonstrated real-time performance at 30 Hz with 10,000 surface points per link, showing that differentiable IK can handle high-resolution geometry while adapting to online tactile measurements.

The integration of \acro{} and Contact-IK on the Boston Dynamics Spot platform validated both components in realistic manipulation scenarios. In contact avoidance mode, the robot successfully reacted to unexpected collisions while maintaining its primary task objective. In contact embrace mode, the system used whole-arm tactile feedback to gently open a door while simultaneously balancing a dustpan in the gripper—demonstrating coordinated multi-objective control that leverages the redundant degrees of freedom available in mobile manipulators.

Full-skin characterization on the 6x5 matrix applied to the Spot arm showed a mean contact threshold of 2.78 N with a 1.29 N standard deviation, reflecting non-uniform pressure distribution on the complex link geometry. Despite this variability, the skin provided reliable contact detection for manipulation tasks, and the fitted per-taxel curves support force estimation during control. Future work could explore improved consistancy across geometries through alternative designs or materials.

Several limitations suggest directions for future work. While the current \acro{} design provides a practical solution, contact near ridges may produce non-uniform force readings due to the more rigid structure. Future iterations could explore alternative ridge geometries or materials to improve consistency across the surface. While our sensor provides analog force measurements, we have not yet explored hysteresis or non-linearity characteristics in detail, as we only fit a simple curve to the loading response. More sophisticated calibration methods or machine learning approaches could enhance force estimation accuracy.

While this work provides a simple assembly process, the wiring complexity increases roughly with the square root of the number of taxels. Future designs could investigate on-board microcontrollers. To further simplify fabrication, exploring fully 3D printed conductive materials could eliminate the need for the manual assembly of discrete FSR components and enable printing the entire skin in a single step. 

Looking forward, we see several promising research directions. Integrating CLOAK with learning-based approaches could enable robots to emulate human-like contact behaviors. Extending the framework to whole-body control for Humanoid robots could enable safer physical human-robot interaction beyond traditional bimanual manipulation. 

In summary, this work demonstrates that combining accessible whole-arm tactile sensing with contact-aware kinematic optimization enables mobile manipulators to safely and effectively exploit environmental contacts during manipulation. By addressing both hardware and algorithmic challenges, we take a step toward robots that can interact with their surroundings as fluidly and adaptively as humans do.



% Extras
% \section{Appendix}
\subsection{Related Works}

\begin{itemize}
        \item \href{https://hiro-group.ronc.one/gentacttoolbox/}{GenTact}: This is the main work we aim to improve upon. 
        \begin{itemize}
                \item Strengths:
                \begin{itemize}
                        \item Low-cost to manufacture: This is 3D printed and uses a simple microcontroller to detect contact.
                        \item No Moving parts: Easy to maintain and use.
                        \item Procedurally generated: User defined density and shape. Adaptable to many use cases
                \end{itemize}
                \item Weaknesses:
                \begin{itemize}
                        \item No force measurement: Only works as a binary as they use capacitive sensing. 
                        \item Must be a human: Non-capacitive objects will not activate the sensor. Only objects that would work with a touchscreen will create a reading for this sensor. 
                        \item Single contact only: The sensor is calibrated to only detect a single contact. Multiple contacts are extremely difficult to model or predict. 
                        \item Needs a multifilament 3D Printer: May take a long time to print, and requires a special filament. 
                \end{itemize}
        \end{itemize}

        \item \href{https://ori.ox.ac.uk/media/10994/01-ral-2021.pdf}{Exploiting Distributed Tactile Sensors to Drive a Robot Arm Through Obstacles}
        \begin{itemize}
                \item A work similar to what we may want to do. This uses "taxels", or tactile pixels to navigate through a cluttered environment.
                \item Strengths:
                \begin{itemize}
                        \item Mathematics: This work has a strong base in the analytical kinematics, with a conditioning of the null space and a rigorous exploration of the control algorithm.  
                        \item Sensors: This has a robust custom PCB based sensor array. 
                \end{itemize}
                \item Weaknesses:
                \begin{itemize}
                        \item Lack of generalization: This focuses on a very narrow problem space, perhaps follow-up works address this. 
                        \item Low sensor coverage: while there are many "taxels", this is still significantly lower than what we propose. 
                        \item Cost: this is still an expensive and complicated system. 
                \end{itemize}
        \end{itemize}

        \item \href{https://www.tum.de/en/news-and-events/all-news/press-releases/details/35732-1}{TUM Robot Skin}
        \begin{itemize}
                \item This uses discrete skin sensor units that has individual force, temp, acceleration, and magnetic sensors for each unit. 
                \item Strengths:
                \begin{itemize}
                        \item Robustness: This design seems both strong and good for large scale coverage. 
                        \item Multimodal: With illumination control and multisensory readings, this has a better observation space than what we propose. 
                        \item Event driven: This uses a unique interface, that we may use. The control framework is extensive.
                \end{itemize}
                \item Weaknesses:
                \begin{itemize}
                        \item Cost: each unit is very expensive compared to our design.
                        \item Density: This does not have the same resolution or adaptability that our design has. Each unit is the same size. 
                        \item Complexity: This has many more features than we would want for our current application. 
                \end{itemize}
        \end{itemize}

        \item \href{https://arxiv.org/pdf/1304.6146}{Manipulation in clutter: the plant one}
        \begin{itemize}
                \item This work focuses on the manipulation in dense foliage and clutter. It has a controller that we can use as a baseline for our experiments. 
        \end{itemize}

        \item \href{https://ox5bc.github.io/public_html/paper/Sonic_skin_final.pdf}{Enabling Low-Cost Full Surface Tactile Skin for Human Robot Interaction}
        \begin{itemize}
                \item This uses Piezo patches and a wireless sensor to detect pressure and contact. It seems interesting and dense, but not very robust or scalable. Requires substantial calibration and environmental control.
        \end{itemize}

        \item \href{https://arxiv.org/pdf/1411.6837v1}{A Flexible and Robust Large Scale Capacitive Tactile System for Robots}
        \begin{itemize}
                \item This work uses a flexible PCB based approach to what we are aiming for. They use capacitive elements, and several groups of cells on each PCB to make a unit. These require their own encasement to work on a robot.      
        \end{itemize}

        \item \href{https://ieeexplore.ieee.org/abstract/document/10068344}{TacSuit: A Wearable Large-Area, Bioinspired Multimodal Tactile Skin for Collaborative Robots}
        \begin{itemize}
                \item TODO
        \end{itemize}

        \item \href{https://ieeexplore.ieee.org/document/9561152}{Human-Like Artificial Skin Sensor for Physical Human-Robot Interaction}
        \begin{itemize}
                \item This work uses a high resolution capacitive array of electrodes in a grid for tactile sensing. They have a complicated hardware assembly process, but it results in a flexible sensing method. They do not have any kinematics or controller associated with this work.
        \end{itemize}

        \item \href{https://ieeexplore.ieee.org/document/10610014}{CushSense: Soft, Stretchable, and Comfortable Tactile-Sensing Skin for Physical Human-Robot Interaction}
        \begin{itemize}
                \item They use a layered approach to the "taxel" robot sensing method, with a high deformation sensor that can be squished. Their controller interface is questionable, but the individual cell design could be interesting. One major takeaway could be that we could sandwich a compressible layer below or above our FSRs.
        \end{itemize}

        \item \href{https://www.khoury.northeastern.edu/home/lsw/papers/icra2024-tactile.pdf}{A Hierarchical Framework for Robot Safety using Whole-body Tactile Sensors} 
        \begin{itemize}
                \item This work proposes a mathematical framework for creating safety for robots with tactile sensing. We may be able to implement this with our loss function. This uses the TUM sensors as a sensing method. 
        \end{itemize}

        \item \href{https://arxiv.org/pdf/2110.14516}{Self-Contained Kinematic Calibration of a Novel Whole-Body Artificial Skin for Human-Robot Collaboration}
        \begin{itemize}
                \item This proposes a calibration pipeline to locate where a tactile cell is on a robot. This can be very useful for us if we have issues with aligning sensors or want to verify our predicted positions. 
        \end{itemize}

        \item \href{https://ieeexplore.ieee.org/document/10341766}{AmbiSense: Acoustic Field Based Blindspot-Free Proximity Detection and Bearing Estimation}
        \begin{itemize}
                \item This uses some sort of single, non contact based proximity sensor to detect and avoid contact. The control algorithms used in this may be helpful. 
        \end{itemize}

        \item \href{https://arxiv.org/pdf/2111.09687}{A Low-Cost, Easy-to-Manufacture, Flexible, Multi-Taxel Tactile Sensor and its Application to In-Hand Object Recognition}
        \begin{itemize}
                \item TODO
                \item Strengths:
                \begin{itemize}
                        \item TODO
                \end{itemize}
                \item Weaknesses:
                \begin{itemize}
                        \item TODO
                \end{itemize}
        \end{itemize}

        \item \href{https://www.science.org/doi/10.1126/scirobotics.adn4008}{Intrinsic sense of touch for intuitive physical human-robot interaction}
        \begin{itemize}
                \item TODO
                \item Strengths:
                \begin{itemize}
                        \item TODO
                \end{itemize}
                \item Weaknesses:
                \begin{itemize}
                        \item TODO
                \end{itemize}
        \end{itemize}


        \item \href{https://link.springer.com/article/10.1007/s10853-023-09091-1}{Review: Progress on 3D printing technology in the preparation of flexible tactile sensors}
        \begin{itemize}
                \item TODO
        \end{itemize}


        \item \href{https://www.sciencedirect.com/science/article/pii/S1002007123000175}{Recent progress in high-resolution tactile sensor array: From sensor fabrication to advanced applications}
        \begin{itemize}
                \item TODO
        \end{itemize}


        \item \href{https://advanced.onlinelibrary.wiley.com/doi/full/10.1002/admt.201900147}{3D-Printed Flexible Tactile Sensor Mimicking the Texture and Sensitivity of Human Skin}
        \begin{itemize}
                \item TODO
        \end{itemize}


        \item \href{https://ieeexplore.ieee.org/document/10303764}{A Computational Design Pipeline to Fabricate Sensing Network Physicalizations}
        \begin{itemize}
                \item TODO
                \item Strengths:
                \begin{itemize}
                        \item TODO
                \end{itemize}
                \item Weaknesses:
                \begin{itemize}
                        \item TODO
                \end{itemize}
        \end{itemize}

        \item \href{https://ieeexplore.ieee.org/document/9592677}{Proximity Perception in Human-Centered Robotics: A Survey on Sensing Systems and Applications}
        \begin{itemize}
                \item TODO
        \end{itemize}

        \item \href{https://dl.acm.org/doi/10.1145/3658185}{Capacitive Touch Sensing on General 3D Surfaces}
        \begin{itemize}
                \item TODO
        \end{itemize}

        \item \href{https://dl.acm.org/doi/10.1145/3332165.3347895}{Multi-Touch Kit: A Do-It-Yourself Technique for Capacitive Multi-Touch Sensing Using a Commodity Microcontroller}
        \begin{itemize}
                \item TODO
        \end{itemize}

        \item \href{https://ww1.microchip.com/downloads/aemDocuments/documents/OTH/ProductDocuments/LegacyCollaterals/00001334B.pdf}{Techniques for Robust Touch Sensing Design}
        \begin{itemize}
                \item TODO
        \end{itemize}

        \item \href{https://arxiv.org/pdf/2310.17274}{cuRobo:Parallelized Collision-Free Minimum-Jerk Robot Motion Generation}
        \begin{itemize}
                \item TODO
        \end{itemize}

        \item \href{https://spj.science.org/doi/10.34133/2019/3018568}{A Review of Printable Flexible and Stretchable Tactile Sensors}
        \begin{itemize}
                \item TODO
        \end{itemize}

\end{itemize}

\subsection{Extras:}
\begin{itemize}
    \item \href{https://docs.isaacsim.omniverse.nvidia.com/latest/assets/usd_assets_robots.html}{NVIDIA Isaac Sim: USD Robot Assets}
    \item \href{https://polytouch.alanz.info/}{Poly Touch: Multimodal sensing}
    \item \href{https://arxiv.org/pdf/2202.02207}{Tactile-Vision Pose estimation}
    \item \href{https://arxiv.org/pdf/2312.02711}{Humanoid motion control with tactile sensing}
    \item \href{https://arxiv.org/pdf/1904.02111}{Dressing HRI with tactile}
    \item \href{https://mediatum.ub.tum.de/doc/1116561/360779.pdf}{TUM's cell based approach hardware description}
    \item \href{https://pubs.acs.org/doi/10.1021/acsnano.2c06432}{Large-Scale Integrated Flexible Tactile Sensor Array for Sensitive Smart Robotic Touch}
    \item \href{https://journals.sagepub.com/doi/full/10.1177/02783649231168954}{Passive and active acoustic sensing for soft pneumatic actuators}
    \item \href{https://www.nature.com/articles/s44172-025-00350-4}{Vision-based tactile sensor design using physically based rendering}
    \item \href{https://www.researchgate.net/publication/353921862_Procedural_Dungeon_Generation_A_Survey}{Procedural Dungeon Generation: A Survey}
    \item \href{https://ieeexplore.ieee.org/document/41470}{Geometric continuity of parametric curves: three equivalent characterizations}
    \item \href{https://www.cs.cmu.edu/~fp/courses/graphics/asst5/catmullRom.pdf}{Catmull-Rom splines}
    \item \href{https://programmer.ie/post/poisson/}{Fast Poisson Disk Sampling in Arbitrary Dimensions}
\end{itemize}

\subsection{Boston Dynamics Documentation}
\begin{itemize}
    \item \href{https://dev.bostondynamics.com/docs/concepts/arm/arm_concepts.html#collision-avoidance}{Boston Dynamics Arm Concepts: Collision Avoidance}
    \item \href{https://support.bostondynamics.com/s/article/Spot-Arm-End-Effector-Payload-Specifications-72058}{Spot Arm End Effector Payload Specifications}
    \item \href{https://support.bostondynamics.com/s/article/Spot-Arm-Specifications-151694}{Spot Arm Specifications}
    \item \href{https://support.bostondynamics.com/s/article/Payload-Mount-Points-and-Dimensions-49963}{Payload Mount Points and Dimensions}
    \item \href{https://support.bostondynamics.com/s/article/How-Spot-Arm-Moves-151690}{How Spot Arm Moves}
    \item \href{https://support.bostondynamics.com/s/article/Manually-Operate-Spot-Arm-151692}{Manually Operate Spot Arm}
\end{itemize}


% \section{Methodology}
\label{sec:methodology}


\subsection{Tactile Sensor Design}
\label{sec:sensor_design}
We design a tactile sensor that can be easily integrated onto the Boston Dynamics Spot robot arm, providing whole-arm tactile sensing with high spatial resolution and sensitivity. The sensor is designed to be low-cost, with a BOM of less than \$100, and easy to manufacture using 3D printing and off-the-shelf components.

The sensor consists of a 3D printed inner shell that snaps onto the arm of the robot, and an outer shell that can deform under contact forces. Between the inner and outer shell, we place a layer of piezoelectric conductive fabric that acts as a force-sensitive resistor (FSR). The inner and outer shells are designed to create a grid of sensing cells, where each cell corresponds to a specific location on the arm. The sensor is connected to a micro-controller that reads the voltage across each cell, allowing us to estimate the contact force and location.

The sensor is designed to be robust to environmental factors, such as sharp objects, dirt, and moisture. The outer shell is made from a 95A shore hardness TPU, which provides a good balance between flexibility and durability. The inner shell is made from a rigid material, such as PLA or ABS, to provide structural support and ensure a secure fit onto the arm. We use a highly conductive copper tape to create the sensor traces, which easily cut using the internal traces in the 3D printed shells. The sensor is designed to be modular, allowing us to easily replace or upgrade individual components as needed. The layers are assembled using a needle and thread, sewing the conductive fabric to the inner shell, and the outer shell is then sewed on top, creating a sealed unit that protects the internal components.

\subsection{Inverse Kinematics with Contact-Aware Conditioning}
\label{sec:contact_ik}
Given a robot arm with $M$ joints with joint angles $\omega$, we define the forward kinematics function as $FK(\omega)$, which returns the position and orientation of the end effector in SE(3). The robot is underactuated if $M < 6$, and overactuated if $M > 6$. An underactuated is not guaranteed to reach every position and orientation in SE(3), while an overactuated robot has infinite solutions to reach a given position and orientation. We focus on the overactuated case, as the Boston Dynamics Spot arm has 6 revolute joints and we add two prismatic joints to control the base of the arm, leading to a total of 8 joints. Because the arm is overactuated, we can condition the inverse kinematics to optimize for other objectives, such as avoiding obstacles or embracing contact. This can be refferred to as conditioning the null space of the inverse kinematics. A null space is the set of all joint angles that do not affect the end effector position and orientation, so in our case of $M=8$ joints, the null space has a dimension of $M-6=2$.

We define the forward kinematics function as $FK(\omega): \mathbb{R}^M \rightarrow SE(3)$, which can be decomposed into a series of transformations from each joint. Our approach is agnostic to the specific forward kinematics implementation, as long as it is differentiable, allowing compatibility with most robot representations. Generally, the forward kinematics can be expressed as a series of matrix multiplications of the individual joint transformations. The forward kinematics can be expressed as:
\begin{equation}
\label{eq:fk}
FK(\omega) = \prod_{i=1}^{M} {}^{i-1}\!T_{i}(\omega_i) = T_{1}(\omega_1) T_{2}(\omega_2) \ldots T_{M}(\omega_M)
\end{equation}

where $T_i(\omega_i)$ is the transformation matrix for joint $i$ given its joint angle $\omega_i$. T can be computed using the Denavit-Hartenberg parameters or other methods, depending on the robot's kinematic structure.The psuedocode for the forward kinematics is shown in Algorithm \ref{alg:conditioning}.

Given a desired end effector position and orientation $X_d \in SE(3)$, we use the quasi-Newton method to iteratively solve for the next joint angles $\omega_{t+1}$ that minimize the error between the current end effector position $X_c=FK(\omega_t)$ and the desired position $X_d$. Our optimization problem is defined as:
\begin{equation}
\label{eq:ik}
\omega_{t+1} = \arg\min_{\omega} \| X_c - X_d \|^2 + \lambda R(\omega)
\end{equation}
where $R(\omega)$ is a conditioning term that encodes our secondary objectives, such as avoiding obstacles or minimizing contact forces, and $\lambda$ is a weighting factor that balances the importance of the primary and secondary objectives.

The optimization is performed continuously, updating the joint angles at each iteration. Our approach does not require an explicit Jacobian matrix or a hessian, as we use automatic differentiation to compute the gradients needed for the quasi-Newton updates. Using the automatic differentiation allows us to easily incorporate complex conditioning terms and constraints into the optimization, as long as they are differentiable.

\subsection{Implementation Details}
We implement the forward kinematics function $FK(\omega)$ using MuJoco's XML format, specifying the joint types, link lengths, and other relevant parameters. We use the \texttt{mujoco-py} library to load the model and compute the forward kinematics using PyTorch tensors to enable automatic differentiation. We define the optimization loop using PyTorch's autograd functionality to compute the gradients of the loss function with respect to the joint angles. We use the TODO optimizer from PyTorch to perform the quasi-Newton updates, as it is well-suited for problems with a moderate number of parameters and can handle non-linear objectives. Our gradient descent loop is shown in Algorithm \ref{alg:ik}.

\begin{algorithm}
\caption{Inverse Kinematics with Contact-Aware Conditioning}\label{alg:ik}
\begin{algorithmic}[1]
\Require Desired end effector pose $X_d$, initial joint angles $\omega_0$, gradient update scalar $\alpha$
\State $\omega \gets \omega_0$
\While {True}
    \State $X_c \gets FK(\omega)$ \Comment{Compute current end effector pose}
    \State $L \gets \| X_c - X_d \|^2 + \lambda R(\omega)$ \Comment{Compute loss with conditioning}
    \State $g \gets \nabla_\omega L$ \Comment{Compute gradient of loss w.r.t. joint angles}
    \State $\omega \gets \omega - \alpha g$ \Comment{Update joint angles}
    % \State \Return $\omega$ \Comment{Return optimized joint angles}
    \State \textbf{wait} until next control cycle
\EndWhile
\end{algorithmic}
\end{algorithm}

Our conditioning term $R(\omega)$ can be tuned according to the desired behavior of the robot. For example, to minimize contact forces, we can define $R(\omega)$ as the sum of squared contact forces sensed by the tactile skin. To avoid obstacles, we can define $R(\omega)$ as a penalty for joint configurations that bring the arm close to contact locations sensed by the tactile skin. The specific form of $R(\omega)$ can be adjusted based on the application and the characteristics of the tactile sensing system. The algorithm for computing the conditioning term is shown in Algorithm \ref{alg:conditioning}. Our penalty function can be a simple linear or quadratic function of the sensor readings, or a more complex function that takes into account the geometry of the robot and the environment. Our penalty function creates a new IK objective that encourages each sensor cell to move away from the contact location as shown in algorithm \ref{alg:penalty}.

        

\begin{algorithm}
\caption{Modified Forward Kinematics and Conditioning Term Computation}\label{alg:conditioning}
\begin{algorithmic}[1]
\Require Current joint angles $\omega$, tactile sensor readings $S$, contact threshold $\text{threshold}$, penalty function $P()$
\State $R \gets 0$
\State $T_0 \gets I$ \Comment{Initialize transformation matrix to identity}
\For{$i = 1$ to $M$}
        \State $T_i \gets {}^{i-1}\!T_{i}(\omega_i)$ \Comment{Compute transformation for joint $i$}
        \State $T_0 \gets T_0 \cdot T_i$ \Comment{Update cumulative transformation}
        \For{each tactile sensor $s$ on link $i$}
            \If{$S[s] > \text{threshold}$} \Comment{Check for contact}
                \State $R \gets R + P(S[s])$ \Comment{Add penalty based on sensor reading}
            \EndIf
        \EndFor
\EndFor
\State \Return $R$ \Comment{Return computed conditioning term, and end effector pose $T$}
\end{algorithmic}
\end{algorithm}


\begin{algorithm}
\caption{Penalty Function for Contact Avoidance}\label{alg:penalty}
\begin{algorithmic}[1]
\Require Sensor reading $s$: force value from tactile sensor $s_f$, location of sensor in world frame $s_p$, normal vector at sensor location $s_n$, scaling factor $\beta$
\If{$f < f_{thresh}$}
    \State \Return 0 \Comment{No penalty if force is below threshold}
\EndIf 
\State $p_{target} \gets s_p - (s_n \cdot s_f \cdot \beta$) \Comment{Compute target position away from contact}
\State \Return $\| p_{target} - s_p \|^2$ \Comment{Return squared distance as penalty}
\end{algorithmic}
\end{algorithm}

    
\subsection{Evaluation}

We evaluate our sensor and contact-aware inverse kinematics across several metrics:
\begin{enumerate}
    \item Contact localization and force intensity estimation
    \item Maximum contact force with the environment for a given trajectory
    \item Ability to address deformable contact vs rigid contact
\end{enumerate}

Metric (1) evaluates the sensor's accuracy in estimating contact locations and forces, a critical aspect to understand how well the tactile skin can inform the IK solver about the environment. 

Metric (2) assesses the real-world performance of the contact-aware IK by measuring the maximum force exerted on the environment during a predefined trajectory with a single obstacle. This metric indicates how effectively the IK can adapt to normal contact scenarios, by evaluating the saftety of the robot's movements and its ability to follow the desired trajectory. 

Metric (3) tests the IK's capability to implicitly handle different types of contact, allowing for deformable objects to be embraced while avoiding excessive forces on rigid objects. This metric is crucial for applications where the robot interacts with compliant objects, such as in agriculture or human-robot interaction. 

For each metric, we compare the performance of the contact-aware IK against a baseline IK that does not utilize tactile feedback. This comparison highlights the benefits of integrating tactile sensing into the IK process, demonstrating improvements in safety, adaptability, and task performance.

% Bibliography
\bibliographystyle{IEEEtran}
\bibliography{references.bib}

\end{document}
