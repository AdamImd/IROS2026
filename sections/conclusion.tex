\section{Conclusion}

This paper introduced \acro{}, a low-cost whole-arm tactile skin, and Contact-IK, a contact-aware inverse kinematics algorithm that enables robots to both avoid and embrace contact during manipulation. By combining a practical sensing solution with differentiable optimization, we demonstrated that mobile manipulators can leverage whole-arm tactile feedback to perform contact-rich tasks in cluttered environments.

The \acro{} sensor addresses key barriers to widespread adoption of whole-arm tactile sensing. Using 3D-printed modules with FSR arrays and fabric surface layers, it provides a high coverage analog sensor with a material cost under \$10 and requiring less than two hours of assembly time per link. Our characterization showed that a 3mm TPU outer shell provides a favorable balance between sensitivity and dynamic range, with sufficient force detection threshold to suppress noise while maintaining responsiveness for manipulation tasks. The design workflow leverages robot mesh files, and enables adaptation to diverse robotic platforms without specialized fabrication equipment or extensive calibration.

Contact-IK complements this hardware by incorporating tactile feedback directly into the kinematic optimization process. Unlike traditional planners that treat contact as a binary constraint, Contact-IK operates in two modes: contact avoidance, where detected contacts generate repulsive terms in the null-space objective, and contact embrace, where the robot actively seeks and regulates contact forces while maintaining end-effector goals. Our experiments demonstrated real-time performance at 30 Hz with 10,000 surface points per link, showing that differentiable IK can handle high-resolution geometry while adapting to online tactile measurements.

The integration of \acro{} and Contact-IK on the Boston Dynamics Spot platform validated both components in realistic manipulation scenarios. In contact avoidance mode, the robot successfully reacted to unexpected collisions while maintaining its primary task objective. In contact embrace mode, the system used whole-arm tactile feedback to gently open a door while simultaneously balancing a dustpan in the gripper—demonstrating coordinated multi-objective control that leverages the redundant degrees of freedom available in mobile manipulators.

Full-skin characterization on the 6x5 matrix applied to the Spot arm showed a mean contact threshold of 2.78 N with a 1.29 N standard deviation, reflecting non-uniform pressure distribution on the complex link geometry. Despite this variability, the skin provided reliable contact detection for manipulation tasks, and the fitted per-taxel curves support force estimation during control. Future work could explore improved consistancy across geometries through alternative designs or materials.

Several limitations suggest directions for future work. While the current \acro{} design provides a practical solution, contact near ridges may produce non-uniform force readings due to the more rigid structure. Future iterations could explore alternative ridge geometries or materials to improve consistency across the surface. While our sensor provides analog force measurements, we have not yet explored hysteresis or non-linearity characteristics in detail, as we only fit a simple curve to the loading response. More sophisticated calibration methods or machine learning approaches could enhance force estimation accuracy.

While this work provides a simple assembly process, the wiring complexity increases roughly with the square root of the number of taxels. Future designs could investigate on-board microcontrollers. To further simplify fabrication, exploring fully 3D printed conductive materials could eliminate the need for the manual assembly of discrete FSR components and enable printing the entire skin in a single step. 

Looking forward, we see several promising research directions. Integrating CLOAK with learning-based approaches could enable robots to emulate human-like contact behaviors. Extending the framework to whole-body control for Humanoid robots could enable safer physical human-robot interaction beyond traditional bimanual manipulation. 

In summary, this work demonstrates that combining accessible whole-arm tactile sensing with contact-aware kinematic optimization enables mobile manipulators to safely and effectively exploit environmental contacts during manipulation. By addressing both hardware and algorithmic challenges, we take a step toward robots that can interact with their surroundings as fluidly and adaptively as humans do.

