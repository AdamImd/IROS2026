\section{Introduction}
Humans naturally exploit whole-arm contact during manipulation, using surfaces to brace, stabilize, or guide movements in cluttered environments. Recent advances in robotic tactile sensing have demonstrated the value of extending sensing beyond grippers to enable safer physical human-robot interaction~\cite{10610014}, whole-body contact awareness~\cite{8812712,kohlbrenner2025gentact}, and contact-rich manipulation strategies~\cite{8593555,10.7551/mitpress/9481.003.0014}. Despite these advances, widespread adoption remains limited: most robotic systems still lack tactile coverage beyond the hands, limiting their ability to safely and effectively exploit environmental contacts as humans do~\cite{6502183,Cyskin}.

While progress has been made, several challenges continue to hinder broader adoption of whole-arm tactile sensing in robotics. Existing research prototypes that extend tactile sensing to the forearm or whole arm~\cite{6502183,Cyskin,kohlbrenner2024machine} often rely on hardware that is costly, fragile, or difficult to replicate, creating a gap between demonstration systems and practical deployment.
% Moreover, even when tactile data is available, integrating it effectively into control algorithms to enhance safety and performance remains an active challenge~\cite{8276298,9616379,ashkanazy2023collision,10375143}. 
Bridging these gaps—developing accessible, robust tactile hardware and control methods that leverage whole-arm contact—is essential for enabling robots to exploit whole-body contact as humans do.

To bridge these gaps, we introduce \acro{}, a low-cost and robust tactile skin that can be readily integrated into diverse robot embodiments. \acro{} builds on \cite{FSR_WTL} and uses 3D-printed inner and outer shells that form a high-coverage array of force-sensitive resistors (FSRs), generated directly from the arm's 3D mesh to conform to complex geometries. To leverage this sensing capability, we develop Contact-IK, a contact-aware inverse kinematics algorithm that incorporates tactile feedback into null-space motion. This enables robots to regulate contact forces during manipulation—avoiding excessive collisions while proactively embracing beneficial contacts. For mobile manipulators with redundant degrees of freedom, Contact-IK exploits the additional null space to simultaneously achieve end-effector goals while managing whole-body contact.

The contributions of this paper are as follows:
\begin{itemize}
    \item A low-cost, whole-arm tactile skin (\acro{}) consisting of modules with high-coverage FSR arrays built with fabric or TPU surface layers, designed directly from robot mesh models to conform to complex arm geometries.
    \item Fabrication guidelines and sensor characterization demonstrating the practical feasibility and performance of the proposed tactile skin design.
    % \item Contact-IK, a contact-aware inverse kinematics algorithm that incorporates tactile feedback into null-space optimization, enabling both contact avoidance and contact embrace strategies.
    \item Experimental validation on a mobile manipulator platform demonstrating \ktxt{TODO}.
\end{itemize}
